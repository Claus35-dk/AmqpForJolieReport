\section{Introduction}
In this thesis we will extend the Jolie programming language with support for the Advanced Message Queuing Protocol while describing the challenges we face.
\subsection{Problem}
\subsubsection{Problem statement}
Service-oriented computing is a paradigm for developing distributed systems, by composing services that communicate using standardized protocols. This makes it extremely decoupled. Jolie is an interpreted service-oriented language.\\
Advanced Message Queue Protocol (AMQP) is a wire-level message-oriented protocol. The protocol not only describes the communication of the protocol, but also describes types down to the bytes to be transferred across the network. This is what makes it wire-level and its implementers are able to transfer objects of different types.\\
The AMQP protocol is becoming more and more widely used. AMQP is used by companies such as Microsoft, VMWare, RedHat and Cisco. Jolie does not support this protocol, which limits its users from using AMQP services.

The main focus of our project will be to implement an extension for Jolie to extend Jolie with support for the message-oriented protocol AMQP.
\subsubsection{Method}
We will study the AMQP standard and other implementations of the protocol. We will study how other protocols are implemented with Jolie, and we will implement AMQP for Jolie.

Our implementation of AMQP will enable the use of the protocol for regular Jolie method calls on services (request-response). We will also look into how to best implement continually enqueue and consume messages without enquement being blocking for the publisher (publish-subscribe).

We will do our implementation with an incremental approach with the following major parts:
\\\textbf{Milestones}
\begin{enumerate}
\item Implement the AMQP type system in accordance with the Jolie type system.
\item Implement the symmetric asynchronous protocol for transfer of messages between processes.
\item Implement the standard, extensible message format defined in the AMQP specification.
\end{enumerate}

\subsection{Challenges}
Doing this project, we encountered some challenges which we will now describe.
\subsubsection{Jolie integration}
\label{subsubsec:Jolie integration}
We are writing an extension to Jolie, and for it to be useful, it will have to work as an integration and work seamlessly with Jolie.

Jolie makes a promise of separation between behavior and deployment. The programmer should be able to change deployment specific things such as the transportation protocol without having to change anything else in the code. This is something our extension will also have to respect and this required some consideration.

Jolie has a number of interfaces for its extensions and protocols to implement. After implementation, the extension is then registered in a manifest-file. Jolie will use the extension by invoking the methods described in the interface. Jolie does not only expect the extension to comply with the interface, but also a certain behavior. We spend a long time observing which methods are invoked when, what Jolie itself will handle and what Jolie expects the extension to handle, and when the extension should hold connections and entities for future use and when can they be discarded.
\subsubsection{Wire-level}
The Advanced Message Queue Protocol is a wire-level protocol. It means that the protocol describes the data format to be sent across the network. This is not likely to comply with the data format used by Jolie and so it may have to be converted. To do this is out of scope for our project. We hope to reuse existing code to handle this or to find some other way to circumvent this.
\subsubsection{Returning procedure calls}
Invoking a service and expecting to get something in return is a perfectly normal thing - a normal behavior for a program. Event-based or message-based systems does not usually follow this behavior. In event-based systems the program usually broadcasts an event describing something that has happened: "I have done this calculation" or "I have discovered this new file" or "Someone just pushed that button". Other programs may then subscribe to these kinds of events and choose to take some action depending on the event notification. In that sort of system you do not know who subscribes to the events you broadcast and you do not expect a response. AMQP is not really suitable for request-response calls, but we will somehow have to support it in order to maintain the Jolie promise of separation of behavior and deployment. Even though we would not recommend using AMQP to invoke returning methods our extension has to support it for use in existing Jolie programs.
\subsubsection{Concurrency}
Jolie is meant for concurrency so of course our implementation has to support this as well. We had to consider when to create new connections and channels and when to reuse existing. Our extension will receive a request and deliver it to Jolie for calling the method. In case of a returning procedure call, Jolie returns the reply from the method to our extension and we faced a challenge in determining the right requester to send this to to. This problem became apparent when handling a lot of requests at once.
\newpage
