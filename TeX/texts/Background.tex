\section{Background}
In this chapter we will give some background knowledge about the technologies we will be working with: The service oriented programming language Jolie and the wire-level Advanced Message Queue Protocol.
\subsection{Service-oriented architectures}
As mention before Jolie is a service oriented programming language, but what does that mean exactly?\\
Service oriented programs does not expose an interface of methods but an interface of service-calls. Each service invoke other services methods through standardized communication protocols. Service-oriented architectures are easy to modularize and they are very scalable. The easy modularization and scalability of service-oriented architectures makes them ideal for distributed systems.
\subsection{Jolie}
Jolie\footnote{\url{http://www.jolie-lang.org}} is an interpreted language running on the Java Virtual Machine. Jolie makes it extremely easy to host a web service and invoke others. Jolie supports a growing number of protocols and one of the ideas of the language is that the code one writes does not need to be changed when one changes the protocol. One simply choose another Jolie supported protocol and the program functions as always.
\subsection{AMQP}
AMQP\footnote{\url{http://www.amqp.org}}
\subsubsection{Queues and Exchanges explained}
\newpage